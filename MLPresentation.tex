\documentclass{beamer}
\usepackage[latin1]{inputenc}

\newcommand {\framedgraphic}[2] {
    \begin{frame}{#1}
        \begin{center}
            \includegraphics[width=\textwidth,height=0.8\textheight,keepaspectratio]{#2}
        \end{center}
    \end{frame}
}

\usetheme{Warsaw}
\title[Automatic differentiation and neural nets]{Automatic
  Differentiation\\Application to Machine Learning}
\author{Dominic Steinitz}
\institute{Kingston University}
\date{15 September 2013}
\begin{document}

\begin{frame}
\titlepage
\end{frame}

\section{Introduction}

\begin{frame}{The Goal}
\begin{itemize}
\item
Wish to fit a model using training data
\item
As an example assume model is a neural network
\item
Need to minmise a cost function: predicted vs. actual
\item
Highly non-linear
\item
Use steepest descent
\item
To do this we need the derivative of the cost function wrt the parameters
\end{itemize}
\end{frame}

\begin{frame}{But Problems \ldots}
\begin{itemize}
\item
Program calculates the non-linear function (not an explicit function)
\item
Function has lots of parameters
\end{itemize}
\end{frame}

\begin{frame}{Possible Solutions}
Could try bumping
$$
\frac{\partial E(\ldots, w, \ldots)}{\partial w} \approx \frac{E(\ldots, w + \epsilon, \ldots) - E(\ldots, w, \ldots)}{\epsilon}
$$
\begin{itemize}
\item
But we have many thousands of parameters
\item
And we are using floating point arithmetic \ldots
\end{frame}

\framedgraphic{Test}{diagrams/13a2bd186a0e123f040da9491fa98684.png}

\framedgraphic{Test}{diagrams/02c0671aa558b88e5ed6f195b22bbd8a.png}

\framedgraphic{Test}{diagrams/ca75393cd25ce951edcd7133da24a2c6.png}

\end{document}
