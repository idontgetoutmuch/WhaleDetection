\documentclass{beamer}
\usepackage[latin1]{inputenc}
\usepackage{listings}
\usepackage{color}

\newcommand {\framedgraphic}[2] {
    \begin{frame}{#1}
        \begin{center}
            \includegraphics[width=\textwidth,height=0.8\textheight,keepaspectratio]{#2}
        \end{center}
    \end{frame}
}

\usetheme{Warsaw}
\title[Automatic differentiation]{Automatic Differentiation\\A Criminally Underused Tool?}
\author{Dominic Steinitz}
\institute{Kingston University}
\date{15 September 2013}
\begin{document}

\begin{frame}
\titlepage
\end{frame}

\section*{Outline}
    \frame{\tableofcontents}

\section{Introduction}

\subsection{What This Talk is About}

\begin{frame}{What This Talk Is About}

  \begin{itemize}
  \item Artificial Neural Networks (ANNs) are a well-established
    Machine Learning (ML) technique.
  \item Traditionally, training ANNs is done using a technique called
    backpropogation.
  \item It turns out that this is just a specialised version of
    Reverse Automatic Differentiation (RAD) and steepest descent.
  \item This talk explains the theory behind RAD and develops an
    implementation in Haskell of Forward Automatic Differentiation
    (FAD) with a brief application to another ML technique:
    regression.
  \end{itemize}

\end{frame}

\section{A Neural Network Refresher}

\subsection{The Model}


\framedgraphic{Example Neural Net}{diagrams/ca75393cd25ce951edcd7133da24a2c6.png}

\begin{frame}[fragile]{The Model}
In symbols
\begin{itemize}
\item $\boldsymbol{x}$ the input
\item $\hat{\boldsymbol{y}}$ the predicted output
\item $\boldsymbol{y}$ the actual output
\item $w^{(k)}$ the weights in the $k$-th layer.
\item $f$ is the transfer function.
\end{itemize}
\end{frame}

\begin{frame}[fragile]{The Layers}
$$
\begin{aligned}
a_i^{(1)}   &= \sum_{j=0}^{N^{(1)}} w_{ij}^{(1)} x_j       & z_i^{(1)} &= f(a_i^{(1)}) \\
a_i^{(2)}   &= \sum_{j=0}^{N^{(2)}} w_{ij}^{(2)} z_j^{(1)} &  \dots     &= \ldots \\
a_i^{(L-1)} &= \sum_{j=0}^{N^{(L-1)}} w_{ij}^{(L-1)} z_j^{(L-2)} & z_j^{(L-1)} &= f(a_j^{(L-1)}) \\
\hat{y}_i  &= \sum_{j=0}^{N^{(L)}} w_{ij}^{(L)} z_j^{(L-1)} \\
\end{aligned}
$$
\end{frame}

\begin{frame}[fragile]{The Cost Function}
Loss function aka cost function

$$
E(\boldsymbol{w}; \boldsymbol{x}, \boldsymbol{y}) = \frac{1}{2}\|(\hat{\boldsymbol{y}} - \boldsymbol{y})\|^2
$$

To fit the model we need to find the weights which minimise the cost
function.
\end{frame}

\subsection{Backpropagation}

\begin{frame}[fragile]{Steepest Descent}
For each weight, we find the derivative of the cost function wrt that
weight and use that to step a small distance (the learning rate) in
the direction which reduces the cost most quickly.

In summary:

$$
w' = w - \gamma\nabla E(w)
$$
\end{frame}

\begin{frame}[fragile]{Derivatives via Error Signals}
In order to perform steepest descent, we need to calculate

$$
\Delta w_{ij} \equiv \frac{\partial E}{\partial w_{ij}}
$$

Applying the chain rule

$$
\Delta w_{ij} =
\frac{\partial E}{\partial w_{ij}} =
\frac{\partial E}{\partial a_i}\frac{\partial a_i}{\partial w_{ij}} =
\delta_j^{(l)}\frac{\partial a_i}{\partial w_{ij}}
$$
Where the error signal $\delta_j^{(l)}$ is defined as

$$
\delta_j^{(l)} \equiv
\frac{\partial E}{\partial a_j^{(l)}}
$$
\end{frame}

\begin{frame}[fragile]{Derivatives via Error Signals}
Since
$$
a_j^{(l)} = \sum_{i=0}^N w_{ij}^{(l)}z_i^{(l-1)}
$$
we have
$$
\frac{\partial a_i^{(l)}}{\partial w_{ij}^{(l)}} =
\frac{\partial}{\partial w_{ij}^{(l)}}\sum_{k=0}^M w_{kj}^{(l)}z_k^{(l-1)} =
z_i^{(l-1)}
$$
Using the definition of the error signal
$$
\Delta w_{ij}^{(l)} =
\frac{\partial E}{\partial w_{ij}^{(l)}} =
\delta_j^{(l)} z_i^{(l-1)}
$$
\end{frame}

\begin{frame}[fragile]{The Backpropagation Algorithm}
\begin{itemize}
\item
Finding the $z_i$ for each layer is straightforward: we start with the
inputs and propagate forward.
\item
In order to find the $\delta_j$ we need
to start with the outputs a propagate backwards.
\item
For the output layer we have (since $\hat{y}_j = a_j$)
$$
\delta_j = \frac{\partial E}{\partial a_j} = \frac{\partial E}{\partial \hat{y}_j} = \frac{\partial}{\partial \hat{y}_j}\bigg(\frac{1}{2}\sum_{i=0}^M (\hat{y}_i - y_i)^2\bigg) = \hat{y}_j - y_j
$$
\end{itemize}
\end{frame}

\begin{frame}[fragile]{Backpropagation for Hidden Layers}
For a hidden layer using the chain rule

$$
\delta_j^{(l-1)} = \frac{\partial E}{\partial a_j^{(l-1)}} =
\sum_k \frac{\partial E}{\partial a_k^{(l)}}\frac{\partial
  a_k^{(l)}}{\partial a_j^{(l-1)}} =
\sum_k \delta_k^{(l)}\frac{\partial a_k^{(l)}}{\partial a_j^{(l-1)}}
$$
\end{frame}

\begin{frame}[fragile]{Backpropagation for Hidden Layers}
We therefore consider in more detail

$$
\frac{\partial a_k^{(l)}}{\partial a_j^{(l-1)}}
$$

We have

$$
a_k^{(l)} = \sum_i w_{ki}^{(l)}z_i^{(l-1)} = \sum_i w_{ki}^{(l)} f(a_i^{(l-1)})
$$

so that

$$
\frac{\partial a_k^{(l)}}{\partial a_j^{(l-1)}} =
\frac{\sum_i w_{ki}^{(l)} f(a_i^{(l-1)})}{\partial a_j^{(l-1)}} =
w_{kj}^{(l)}\,f'(a_j^{(l-1)})
$$
\end{frame}

\begin{frame}[fragile]{Backpropagation for Hidden Layers}
and thus

$$
\begin{aligned}
\delta_j^{(l-1)} &=
\sum_k \delta_k^{(l)}\frac{\partial
  a_k^{(l)}}{\partial a_j^{(l-1)}} \\
&= \sum_k \delta_k^{(l)} w_{kj}^{(l)}\, f'(a_j^{(l-1)}) \\
&= f'(a_j^{(l-1)}) \sum_k \delta_k^{(l)} w_{kj}^{(l)}
\end{aligned}
$$
\end{frame}

\begin{frame}[fragile]{Backpropagation in Full}
Summarising

\begin{enumerate}
\item
We calculate all $a_j$ and $z_j$ for each layer starting with the
input layer and propagating forward.
\item
We evaluate $\delta_j^{(L)}$ in the output layer using $\delta_j = \hat{y}_j - y_j$.
\item
We evaluate $\delta_j$ in each layer using $\delta_j^{(l-1)} =
f'(a_j^{(l-1)})\sum_k \delta_k^{(l)} w_{kj}^{(l)}$ starting with the output
layer and propagating backwards.
\item
Use $\partial E / \partial w_{ij}^{(l)} = \delta_j^{(l)} z_i^{(l-1)}$ to obtain the
required derivatives in each layer.
\end{enumerate}
\end{frame}

\section{Not Sure Yet}

\begin{frame}[fragile]{ANN: The Classical Approach}
Rather than using bumping or symbolic differentation, for ANN's we can
use backpropagation, a technique which seems to be taught on many
Machine Learning courses.
\end{frame}

\begin{frame}{The Goal}

  \begin{itemize}
  \item Wish to fit a model using training data
  \item As an example assume model is a neural network
  \item Need to minimise a cost function: predicted vs. actual
  \item Highly non-linear
  \item Use steepest descent
  \item To do this we need the derivative of the cost function wrt the
    parameters
\end{itemize}

\end{frame}

\begin{frame}{But Problems \ldots}
\begin{itemize}
\item
Program calculates the non-linear function (not an explicit function)
\item
Function has lots of parameters
\end{itemize}
\end{frame}

\section{Differentiation Methods}

\subsection{Bumping}

\begin{frame}{Bumping}
Could try bumping
$$
\frac{\partial E(\ldots, w, \ldots)}{\partial w} \approx \frac{E(\ldots, w + \epsilon, \ldots) - E(\ldots, w, \ldots)}{\epsilon}
$$
\begin{itemize}
\item
But we have many thousands of parameters
\item
And we are using floating point arithmetic \ldots
\end{itemize}
\end{frame}

\framedgraphic{Floating Point Errors}{diagrams/13a2bd186a0e123f040da9491fa98684.png}

\subsection{Symbolic Differentiation}
\begin{frame}[fragile]{Symbolic Differentiation}
\begin{itemize}
\item
Turn program into mathematical function
\item
Compute the differential also as a function either by hand or using
symbolic differentiation package.
\end{itemize}

But \ldots
\end{frame}

\begin{frame}[fragile]{Python Example}
\ldots consider the following Python fragment
\begin{lstlisting}[language=Python]
import numpy as np

def many_sines(x):
    y = x
    for i in range(1,7):
        y = np.sin(x+y)
    return y
\end{lstlisting}

When we unroll the loop we are actually evaluating

$$
f(x) = \sin(x + \sin(x + \sin(x + \sin(x + \sin(x + \sin(x + x))))))
$$

\end{frame}

\begin{frame}[fragile]{Blow Up}
Now suppose we want to get the differential of this
function. Symbolically this would be

$$
\begin{aligned}
f'(x) &=           (((((2\cdot \cos(2x)+1)\cdot \\
      &\phantom{=} \cos(\sin(2x)+x)+1)\cdot \\
      &\phantom{=} \cos(\sin(\sin(2x)+x)+x)+1)\cdot \\
      &\phantom{=} \cos(\sin(\sin(\sin(2x)+x)+x)+x)+1)\cdot \\
      &\phantom{=} \cos(\sin(\sin(\sin(\sin(2x)+x)+x)+x)+x)+1)\cdot \\
      &\phantom{=} \cos(\sin(\sin(\sin(\sin(\sin(2x)+x)+x)+x)+x)+x)
\end{aligned}
$$
\end{frame}

\begin{frame}[fragile]{Implications for ANN}
\begin{itemize}
\item The mathematical function for an ANN could easily be more
  complicated.
\item Creating the derivative by hand is error prone.
\item To use a symbolic differentiation package (usually) requires
  some manipulation of the original program.
\end{itemize}
\end{frame}

\section{Chain Rule Refresher}

\begin{frame}[fragile]{The Rule}
\begin{itemize}
\item
In Lagrange's notation
$$
(g \circ f)'(a) = g'(f(a))\cdot f'(a)
$$
\item
In Leibniz' notation
$$
\frac{\mathrm{d} (g \circ f)}{\mathrm{d} x}(a) =
\frac{\mathrm{d} g}{\mathrm{d} y}(f(a)) \frac{\mathrm{d} f}{\mathrm{d} x}(a)
$$
where $y = f(x)$.
\item
More suggestively, setting $h = g \circ f$, we can write
$$
\frac{\mathrm{d} h}{\mathrm{d} x} =
\frac{\mathrm{d} h}{\mathrm{d} y} \frac{\mathrm{d} y}{\mathrm{d} x}
$$
where it is understood that $\mathrm{d} h / \mathrm{d} x$ and
$\mathrm{d} y / \mathrm{d} x$ are evaluated at $a$ and $\mathrm{d} h /
\mathrm{d} y$ is evaluated at $f(a)$.
\end{itemize}
\end{frame}

\begin{frame}[fragile]{Example}
Recall that if $f(x) = x^2$ then $f'(x) = 2x$. Suppose we only know
this and we also know the chain rule then we can use this to calculate
the derivative of $f(x) = x^4$.

We have
\begin{itemize}
\item
$f(x) = x^2$ and $g(y) = y^2$ thus $(g \circ f)(x) = x^4$
\item
$f'(x) = 2x$ and $g'(y) = 2y$
\item
The Chain Rule $(g \circ f)'(x) = g'(f(x))f'(x)$
\item
Thus $(g \circ f)'(x) = 2f(x)2x = 2x^22x = 4x^3$
\end{itemize}
\end{frame}


\section{Automatic Differentiation}

\begin{frame}[fragile]{Hard Work --- Can We Generalise}
\begin{itemize}
\item
Phew!
\item
But maybe following P\'{o}lya we can find a problem that is more
general and use this not just for ANNs but for estimating parameters
for other Machine Learning models.
\item
Idea: we drew the Neural Net as a graph maybe if we can draw a program as a
graph we can calculate the derivative of the function which the
program represents.
\end{itemize}
\end{frame}

\subsection{Reverse Mode}

\begin{frame}[fragile]{A Simpler Example}
Consider the function

$$
f(x) = \exp(\exp(x) + (\exp(x))^2) + \sin(\exp(x) + (\exp(x))^2)
$$

We can write this as a data flow graph...
\end{frame}

\framedgraphic{Data Flow Graph}{diagrams/02c0671aa558b88e5ed6f195b22bbd8a.png}

\begin{frame}[fragile]{Sequence of Functions}
We can thus re-write our function as a sequence of simpler functions
in which each function only depends on variables earlier in the
sequence.

$$
\begin{aligned}
u_7    &= f_7(u_6, u_5, u_4, u_3, u_2, u_1) \\
u_6    &= f_6(u_5, u_4, u_3, u_2, u_1) \\
\ldots &= \ldots \\
u_2    &= f_2(u_1)
\end{aligned}
$$
\end{frame}

\begin{frame}[fragile]{Differentials}
$$
\begin{aligned}
\mathrm{d}u_7    &= \frac{\partial f_7}{\partial u_6} \mathrm{d} u_6 +
                    \frac{\partial f_7}{\partial u_5} \mathrm{d} u_5 +
                    \frac{\partial f_7}{\partial u_4} \mathrm{d} u_4 +
                    \frac{\partial f_7}{\partial u_3} \mathrm{d} u_3 +
                    \frac{\partial f_7}{\partial u_2} \mathrm{d} u_2 +
                    \frac{\partial f_7}{\partial u_1} \mathrm{d} u_1 \\
\mathrm{d}u_6    &= \frac{\partial f_6}{\partial u_5} \mathrm{d} u_5 +
                    \frac{\partial f_6}{\partial u_4} \mathrm{d} u_4 +
                    \frac{\partial f_6}{\partial u_3} \mathrm{d} u_3 +
                    \frac{\partial f_6}{\partial u_2} \mathrm{d} u_2 +
                    \frac{\partial f_6}{\partial u_1} \mathrm{d} u_1 \\
\ldots           &= \ldots \\
\mathrm{d}u_2    &= \frac{\partial f_2}{\partial u_1} \mathrm{d} u_1
\end{aligned}
$$
\end{frame}

\begin{frame}[fragile]{Differentials Example}
In our particular example, since $u_1, \dots, u_5$ do not depend on $u_6$

$$
\begin{aligned}
\frac{\partial u_7}{\partial u_6} &= 1
\end{aligned}
$$

Further $u_6$ does not depend on $u_5$ so we also have

$$
\begin{aligned}
\frac{\partial u_7}{\partial u_5} &= 1 \\
\end{aligned}
$$
\end{frame}

\begin{frame}[fragile]{Differentials Example More Interesting}
Now things become more interesting as $u_6$ and $u_5$ both depend on
$u_4$.

$$
\begin{aligned}
\frac{\partial u_7}{\partial u_4} &=
 \frac{\partial u_7}{\partial u_6}\frac{\partial u_6}{\partial u_4} +
 \frac{\partial u_7}{\partial u_5}\frac{\partial u_5}{\partial u_4} \\
&= \frac{\partial u_7}{\partial u_6}\exp{u_4} +
   \frac{\partial u_7}{\partial u_5}\cos{u_5}
\end{aligned}
$$
\end{frame}

\begin{frame}[fragile]{Differentials Example Carrying On}
$$
\begin{aligned}
\frac{\partial u_7}{\partial u_3} &=
 \frac{\partial u_7}{\partial u_4}\frac{\partial u_4}{\partial u_3} \\
&= \frac{\partial u_7}{\partial u_4} \\
\frac{\partial u_7}{\partial u_2} &=
 \frac{\partial u_7}{\partial u_4}\frac{\partial u_4}{\partial u_2} +
 \frac{\partial u_7}{\partial u_3}\frac{\partial u_3}{\partial u_2} \\
&= \frac{\partial u_7}{\partial u_4} + 2u_2\frac{\partial u_7}{\partial u_4} \\
\frac{\partial u_7}{\partial u_1} &=
 \frac{\partial u_7}{\partial u_2}\frac{\partial u_2}{\partial u_1} \\
&=\frac{\partial u_7}{\partial u_2}\exp{u_2}
\end{aligned}
$$
\end{frame}

\begin{frame}[fragile]{Reverse Mode Conclusions}
\begin{itemize}
\item
Note that having worked from top to bottom (the forward sweep) in the
graph to calculate the function itself, we have to work backwards from
bottom to top (the backward sweep) to calculate the derivative.
\item
So provided we can translate our program into a data flow graph, we can
apply this procedure to calculate the differential with the same
complexity as the original program.
\item
The pictorial representation of an ANN is effectively the data flow
graph of the cost function (without the final cost calculation itself)
and its differential can be calculated as just being identical to
backpropagation.
\end{itemize}
\end{frame}

\subsection{Forward Mode}

\begin{frame}[fragile]{Dual Numbers}
\begin{itemize}
\item
An alternative method for automatic differentiation is called forward
mode and has a simple implementation. Let us illustrate this using
Haskell. The actual implementation is about 20 lines of code.
\item
Let us define dual numbers

\begin{lstlisting}[language=Haskell]
data Dual = Dual Double Double
  deriving (Eq, Show)
\end{lstlisting}
\item
We can think of these pairs as first order polynomials in the
indeterminate $\epsilon$, $x + \epsilon x'$ such that $\epsilon^2 = 0$
\end{itemize}
\end{frame}

\begin{frame}[fragile]{Dual Numbers}
Thus, for example, we have

$$
\begin{aligned}
(x + \epsilon x') + (y + \epsilon y') &= ((x + y) + \epsilon (x' + y')) \\
(x + \epsilon x')(y + \epsilon y') &= xy + \epsilon (xy' + x'y) \\
\log (x + \epsilon x') &=
\log x (1 + \epsilon \frac {x'}{x}) =
\log x + \epsilon\frac{x'}{x} \\
\sqrt{(x + \epsilon x')} &=
\sqrt{x(1 + \epsilon\frac{x'}{x})} =
\sqrt{x}(1 + \epsilon\frac{1}{2}\frac{x'}{x}) =
\sqrt{x} + \epsilon\frac{1}{2}\frac{x'}{\sqrt{x}} \\
\ldots &= \ldots
\end{aligned}
$$
\end{frame}

\begin{frame}[fragile]{Dual Numbers and The Chain Rule}
Notice that these equations implicitly encode the chain rule. For
example, we know, using the chain rule, that

$$
\frac{\mathrm{d}}{\mathrm{d} x}\log(\sqrt x) =
\frac{1}{\sqrt x}\frac{1}{2}x^{-1/2} =
\frac{1}{2x}
$$

And using the example equations above we have

$$
\begin{aligned}
\log(\sqrt {x + \epsilon x'}) &= \log (\sqrt{x} + \epsilon\frac{1}{2}\frac{x'}{\sqrt{x}}) \\
                              &= \log (\sqrt{x}) + \epsilon\frac{\frac{1}{2}\frac{x'}{\sqrt{x}}}{\sqrt{x}} \\
                              &= \log (\sqrt{x}) + \epsilon x'\frac{1}{2x}
\end{aligned}
$$
\end{frame}

\begin{frame}[fragile]{Dual Numbers Example}
Notice that dual numbers carry around the calculation and the
derivative of the calculation.

\begin{itemize}
\item
To actually evaluate $\log(\sqrt{x})$
at a particular value, say 2, we plug in 2 for $x$ and 1 for $x'$
$$
\log (\sqrt{2 + \epsilon 1}) = \log(\sqrt{2}) + \epsilon\frac{1}{4}
$$
\item
Thus the derivative of $\log(\sqrt{x})$ at 2 is $1/4$.
\end{itemize}
\end{frame}

\begin{frame}[fragile]{Dual Numbers are Numbers}
\begin{scriptsize}
\begin{lstlisting}[language=Haskell]
instance Num Dual where
  fromInteger n             = undefined
  (Dual x x') + (Dual y y') = Dual (x + y) (x' + y')
  (Dual x x') * (Dual y y') = Dual (x * y) (x * y' + y * x')
  negate (Dual x x')        = Dual (negate x) (negate x')
  signum _                  = undefined
  abs _                     = undefined
\end{lstlisting}
\end{scriptsize}
\end{frame}

\begin{frame}[fragile]{Dual Numbers Can Be Divided}
\begin{scriptsize}
\begin{lstlisting}[language=Haskell]
instance Fractional Dual where
  fromRational p    = undefined
  recip (Dual x x') = Dual (1.0 / x) (-x' / (x * x))
\end{lstlisting}
\end{scriptsize}
\end{frame}

\begin{frame}[fragile]{Dual Numbers are Quasi Floating Point}
\begin{scriptsize}
\begin{lstlisting}[language=Haskell]
instance Floating Dual where
  pi = constD pi
  exp   (Dual x x') = Dual (exp x)   (x' * exp x)
  log   (Dual x x') = Dual (log x)   (x' / x)
  sqrt  (Dual x x') = Dual (sqrt x)  (x' / (2 * sqrt x))
  sin   (Dual x x') = Dual (sin x)   (x' * cos x)
  cos   (Dual x x') = Dual (cos x)   (x' * (- sin x))
  sinh  (Dual x x') = Dual (sinh x)  (x' * cosh x)
  cosh  (Dual x x') = Dual (cosh x)  (x' * sinh x)
  asin  (Dual x x') = Dual (asin x)  (x' / sqrt (1 - x*x))
  acos  (Dual x x') = Dual (acos x)  (x' / (-sqrt (1 - x*x)))
  atan  (Dual x x') = Dual (atan x)  (x' / (1 + x*x))
  asinh (Dual x x') = Dual (asinh x) (x' / sqrt (1 + x*x))
  acosh (Dual x x') = Dual (acosh x) (x' / (sqrt (x*x - 1)))
  atanh (Dual x x') = Dual (atanh x) (x' / (1 - x*x))
\end{lstlisting}
\end{scriptsize}
\end{frame}


\begin{frame}[fragile]{Example}
That's all we need to do. Let us implement the function we considered
earlier.

\begin{scriptsize}
\begin{lstlisting}[language=Haskell]
f = h . h where h x = x * x
\end{lstlisting}
\end{scriptsize}

\begin{scriptsize}
\begin{lstlisting}
ghci> f (Dual 2 1)
  Dual 16.0 32.0
\end{lstlisting}
\end{scriptsize}
\end{frame}

\begin{frame}[fragile]{Automatic Differentiation is {\em Not} Symbolic
    Differentiation}
To see that we are *not* doing symbolic differentiation (it's easy to
see we are not doing numerical differentiation) let us step
through the actual evaluation.

$$
\begin{aligned}
f (\mathrm{Dual}\,2\,1) &\longrightarrow (\lambda z \rightarrow z
\times z \cdot \lambda z \rightarrow z \times z) (\mathrm{Dual}\,2\,1) \\
&\longrightarrow (\lambda z \rightarrow z
\times z) (\mathrm{Dual}\,(2 \times 2)\,(2 \times 1 + 2 \times 1)) \\
&\longrightarrow  (\lambda z \rightarrow z
\times z) (\mathrm{Dual}\,4\,4)\\
&\longrightarrow (\mathrm{Dual}\,(4 \times 4)\,(4 \times 4 + 4 \times 4)) \\
&\longrightarrow (\mathrm{Dual}\,16\,32)\\
\end{aligned}
$$
\end{frame}

\section{Application}

\begin{frame}[fragile]{Linear Regression}
$$
L(\boldsymbol{x}, \boldsymbol{y}, m, c) = \frac{1}{2n}\sum_{i=1}^n (y - (mx + c))^2
$$
\begin{scriptsize}
\begin{lstlisting}[language=Haskell]
cost m c xs ys = (sum (zipWith errSq xs ys)) /
                 (2 * (fromIntegral (length xs)))
  where
    errSq x y = z * z
      where
        z = y - (m * x + c)
\end{lstlisting}
\end{scriptsize}
\end{frame}

\begin{frame}[fragile]{Linear Regression}
\begin{scriptsize}
\begin{lstlisting}[language=Haskell]
g m c = cost m c xs ys

zs = (0.1, 0.1) : map f zs
  where

    deriv (Dual _ x') = x'

    f (c, m) = (c - gamma * cDeriv, m - gamma * mDeriv)
      where
        cDeriv = deriv (g (Dual m 0) (Dual c 1))
        mDeriv = deriv (flip g (Dual c 0) (Dual m 1))
\end{lstlisting}
\end{scriptsize}
\end{frame}

\begin{frame}[fragile]{Linear Regression}
\begin{scriptsize}
\begin{lstlisting}
ghci> take 10 zs
  [(0.1,0.1),
   (0.554,3.2240000000000006),
   (0.30255999999999983,1.4371599999999995),
   (0.4542824,2.4573704000000003),
   (0.3754896159999999,1.8730778559999997),
   (0.42839290304,2.2059302422400004),
   (0.4059525336255999,2.0145512305216),
   (0.42651316156582386,2.1228327781207037),
   (0.42242942391663607,2.059837404270339),
   (0.43236801802049607,2.0947533284323567)]
\end{lstlisting}
\end{scriptsize}
\end{frame}

\section{Concluding Thoughts}

\begin{frame}[fragile]{Efficiency}
Perhaps AD is underused because of efficiency?

It seems that the Financial Services industry is aware that AD is {\em
  more} efficient than current practice. Order of magnitude
improvements have been reported.
\begin{itemize}
\item
Smoking Adjoints: fast evaluation of Greeks in Monte Carlo Calculations
\item
Adjoints and automatic (algorithmic) differentiation in computational finance
\end{itemize}

Perhaps AD is slowly permeating into Machine Learning as well but
there seem to be no easy to find benchmarks.

\begin{itemize}
\item
Good Morning, Economics Blog
\item
Andrew Gelman's Blog
\end{itemize}
\end{frame}

\begin{frame}[fragile]{Tools}
  If it were only possible to implement automatic differentiation in
  Haskell then its applicability would be somewhat
  limited. Fortunately this is not the case and it can be used in many
  languages. In general, there are three different approaches:

\begin{itemize}
\item Operator overloading: available for Haskell and C++. See the
  Haskell ad package and the C++ FADBAD approach using templates.
\item Source to source translators: available for Fortran, C and other
  languages e.g., ADIFOR, TAPENADE and see the wikipedia entry for a
  more comprehensive list.
\item New languages with built-in AD primitives.
\end{itemize}
\end{frame}

\begin{frame}[fragile]{Resources}
\begin{itemize}
\item \textcolor{blue}{http://idontgetoutmuch.wordpress.com}
\item
  \textcolor{blue}{http://en.wikipedia.org/wiki/Automatic\_differentiation}
\item \textcolor{blue}{http://www.autodiff.org}
\end{itemize}
\end{frame}

\end{document}
